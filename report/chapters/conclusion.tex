\chapter{Conclusion}
In this project, we successfully implemented a custom modulation scheme and packet format for the LilyGo T-TWR to enable digital message transmission. By integrating Binary Frequency Shift Keying (BFSK) modulation, Barker codes for synchronization, and Hamming codes for error correction, we were able to establish a functional communication system between two devices. Despite the limitations of our hardware, we succeeded in implementing a reliable two-way message exchange where each LilyGo T-TWR is sender as well as receiver.
\\ \\
During this project we gained insight into digital communication techniques and the challenges associated with wireless signal processing. We encountered and overcame several obstacles, including memory constraints, the need for an efficient demodulation method, and the challenge of achieving a balance between transmission speed and error resilience. Through further optimization we managed to improve the transmission rate from 4 Baud to 16 Baud while still maintaining accurate message reception.
\\ \\
However, certain limitations remain. Our system only supports BFSK modulation, and we rely on zero-crossing detection rather than more advanced demodulation techniques like FFT. Additionally, transmission speeds remain relatively slow at 1 byte per second due to error correction, which could be improved with further optimization. Finally, while our implementation works well for audio-based transmission, an extension to actual radio wave transmission would be a valuable next step.
\\ \\
Future work could explore alternative modulation techniques, more efficient error correction methods, and an expansion to support higher data rates. Despite these limitations, we achieved our goal of creating a communication system between the LilyGo T-TWR devices with our own packet format and custom modulation scheme.