\chapter{Introduction}
{NOTE: Not sure what to say here. I'd just give an overview of how we tackled the problem.
Describing the LILYGO device would be a good thing to do here instead of in Execution.

Alternative:
Explain everything about process in process and here, explain all the tools and methods used, like barker codes, mod/demod etc., assuming readers are not limited to seminar participants -> I think this might be appreciated by Tschudin}





The initial goal of this project was to set up radio communication between two Arduino T-TWR development boards completely from scratch to transmit data in the radio frequency spectrum. After intense trial and error, we discovered that directly accessing and interacting with the radio frequency chip on the board was not possible.\\

After recognizing this limitation in close communication with Prof. Dr. Tschudin, we decided to try transmitting data over the audio frequency range.

The first step was to achieve a working connection between the two devices. This turned turned out to be a challenge, since there was a considerable amount of noise. To mitigate the noise and make sure a clean signal could be received, we used [...]. This solved the problem elegantly and made the connection between both devices more reliable.

\section{Tools}
[Add photos of lilygo and describe lilygo]
Take photos from here: https://lilygo.cc/products/t-twr-rev2-1


\section{Methods}
\subsection{Modulation and Demodulation}
\subsection{Barker Codes}
