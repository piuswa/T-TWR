\chapter{Introduction}
Reliable wireless communication is a key component of modern digital systems, enabling data transfer across devices without physical connections. In this project, we developed a custom modulation scheme and packet format for the LilyGo T-TWR, a development board with built-in Walkie-Talkie functionality, WiFi, Bluetooth, GPS and a SA868 Wireless Transceiver. This work was carried out as part of the Radio Packet Networks seminar at the University of Basel in the HS24.
\\\\
The goal of the project was to establish two-way digital communication between LilyGo T-TWR devices by designing and implementing a structured message format and modulation technique. Sending and receiving text messages through the LilyGo T-TWR is not available by default, only a Walkie-Talkie functionality. The project was divided into two main focus areas. First, a packet format needs to designed which enables reliable message transmission by ensuring correct synchronization and error correction. For synchronization purposes we employ \textit{Barker codes} and for error correction \textit{Hamming codes}. Our second main focus is modulation and demodulation. Both need to be correctly implemented such that binary data can be transmitted and accurately received. We use \textit{Frequency Shift Keying (FSK)} as our modulation method.
\\ \\
By integrating these components, we successfully built a communication system where users can send and receive messages between two devices. This project allowed us to explore real-world signal processing and showed us how much can still be done with limited hardware.
\\ \\
We start this report by first introduction the theoretical background needed for this project. Next, we introduce the LilyGo T-TWR in further detail as well as other tools used in this project. We then explain our implementation in more detail. Finally, we discuss what worked and what caused problems as well as what future work could be done.